\documentclass{article}

\setlength{\headsep}{0.75 in}
\setlength{\parindent}{0 in}
\setlength{\parskip}{0.1 in}

%=====================================================
% Add PACKAGES Here (You typically would not need to):
%=====================================================

\usepackage[margin=1in]{geometry}
\usepackage{amsmath,amsthm}
\usepackage{fancyhdr}
\usepackage{enumitem}
\usepackage{graphicx}

%=====================================================
% Ignore This Part (But Do NOT Delete It:)
%=====================================================

\theoremstyle{definition}
\newtheorem{problem}{Problem}
\newtheorem*{fun}{Fun with Algorithms}
\newtheorem*{challenge}{Challenge Yourself}
\def\fline{\rule{0.75\linewidth}{0.5pt}}
\newcommand{\finishline}{\vspace{-15pt}\begin{center}\fline\end{center}}
\newtheorem*{solution*}{Solution}
\newenvironment{solution}{\begin{solution*}}{{\finishline} \end{solution*}}
\newcommand{\grade}[1]{\hfill{\textbf{($\mathbf{#1}$ points)}}}
\newcommand{\thisdate}{\today}
\newcommand{\thissemester}{\textbf{Rutgers: Fall 2019}}
\newcommand{\thiscourse}{CS 344: Design and Analysis of Computer Algorithms} 
\newcommand{\thishomework}{Number} 
\newcommand{\thisname}{Name} 
\newcommand{\thisextension}{Yes/No} 

\headheight 40pt              
\headsep 20pt
\renewcommand{\headrulewidth}{0pt}
\lhead{\small \textbf{Only for the personal use of students registered in CS 344, Fall 2019 at Rutgers University. Redistribution out of this class is strictly prohibited.}}
\pagestyle{fancy}

\newcommand{\thisheading}{
   \noindent
   \begin{center}
   \framebox{
      \vbox{\vspace{2mm}
    \hbox to 6.28in { \textbf{\thiscourse \hfill \thissemester} }
       \vspace{4mm}
       \hbox to 6.28in { {\Large \hfill Homework \#\thishomework \hfill} }
       \vspace{2mm}
         \hbox to 6.28in { { \hfill \thisdate \hfill} }
       \vspace{2mm}
       \hbox to 6.28in { \emph{Name: \thisname \hfill Extension: \thisextension}}
      \vspace{2mm}}
      }
   \end{center}
   \bigskip
}

%=====================================================
% Some useful MACROS (you can define your own in the same exact way also)
%=====================================================


\newcommand{\ceil}[1]{{\left\lceil{#1}\right\rceil}}
\newcommand{\floor}[1]{{\left\lfloor{#1}\right\rfloor}}
\newcommand{\prob}[1]{\Pr\paren{#1}}
\newcommand{\expect}[1]{\Exp\bracket{#1}}
\newcommand{\var}[1]{\textnormal{Var}\bracket{#1}}
\newcommand{\set}[1]{\ensuremath{\left\{ #1 \right\}}}
\newcommand{\poly}{\mbox{\rm poly}}


%=====================================================
% Fill Out This Part With Your Own Information:
%=====================================================


\renewcommand{\thishomework}{3} %Homework number
\renewcommand{\thisname}{FIRST LAST} % Your name
\renewcommand{\thisextension}{Yes/No} % Pick only one of the two options accordingly

\begin{document}

\thisheading


\begin{problem}
	Recall that in the knapsack problem, we are given $n$ items with positive integer weights $w_1,\ldots,w_n$ and values $v_1,\ldots,v_n$, and a knapsack of size $W$; we want to pick a subset of items with maximum total value 
	that fit the knapsack, i.e., their total weight is not larger than the size of the knapsack. In the class, we designed a dynamic programming algorithm for this problem with $O(nW)$ runtime. Our goal in this problem is to design a different 
	dynamic programming solution. 
	
	Suppose you are told that the \emph{value} of each item is a \emph{positive integer} between $1$ and some integer $V$. Design a dynamic programming algorithm for this problem with worst case $O(n^2 \cdot V)$ runtime. 
Note that there is no restriction on the value of $W$ in this problem. \grade{25}
 
\end{problem}

\begin{solution}
	Solution to problem one goes here. 
\end{solution}

\smallskip

\begin{problem}
	You are given a set of $n$ boxes with dimensions specified in the (length, width, height) format. A box $i$ fits into another box $j$ if \emph{every} dimension of the box $i$ is \emph{strictly smaller} than 
	the corresponding dimension of the box $j$. Your goal is to determine the \emph{maximum length} of a sequence of boxes so that each box in the sequence can fit into the next one. 
	
	Design an $O(n^2)$ time dynamic programming algorithm that given the  arrays $L[1:n]$, $W[1:n]$, and $H[1:n]$, where for every $1 \leq i \leq n$, 
	$(L[i],W[i],H[i])$ denotes the (length, width, height) of the $i$-th box, outputs the length of the longest sequence of boxes $i_1,\ldots,i_k$ so that box $i_1$ can fit into box $i_2$, $i_2$ can fit into $i_3$, and so on and so forth.  
	Note that a box $i$ can fit into another box $j$ if $L[i] < L[j]$, $W[i] < W[j]$, and $H[i] < H[j]$ (you are \emph{not} allowed to rotate any box).  \grade{25}
	
\end{problem}

\begin{solution}
	Solution to problem two goes here. 
\end{solution}

\smallskip

\begin{problem}
	You have a bag of $m$ cookies and a group of $n$ friends. For each of your friends, you know the ``greed factor'' of your friend (denoted by $g_i$ for your $i$-th friend): 
	this is the minimum number of cookies you should give to this friend to make them stop complaining. Of course, you would like to find a way to distribute your cookies in a way to minimize the number of your friends that are still complaining. 
	
	Design an $O(m+n)$ time greedy algorithm to find an assignment of the cookies to your friends so as the minimize the number of the friends that are still complaining: 
	recall that a friend $i$ stops complaining if we assign them $c_i$ cookies and $c_i \geq g_i$. \grade{25}
	
\end{problem}

\begin{solution}
	Solution to problem three goes here. 
\end{solution}

\smallskip

\begin{problem}\label{billboard}
	There is a straight highway with $n$ houses alongside it. You have developed a brilliant new product and would like to advertise it by placing billboards alongside this highway. 
	Since constructing billboards is expensive, you would like to choose as few billboard locations as possible such that every house is within at most $d$ units of a billboard. 
	 
	Design an $O(n\log{n})$ time greedy algorithm which given an array $A$ of real numbers representing the locations of the houses alongside the highway (house one is placed on location $A[1]$, house two on $A[2]$, etc.), and a maximum distance $d$,
  	outputs a \emph{minimum} set of locations (i.e., real numbers) for placing the billboards such that each house is at most $d$ units from some billboard. \grade{25}
		
\end{problem}

\begin{solution}
	Solution to problem four goes here. 
\end{solution}


\smallskip

\begin{challenge}
	Let us revisit Problem~\ref{billboard}. Instead of minimizing the number of billboards so that every house is within distance $d$ of some billboard, we now have a budget of $b$ billboards in total. Our goal is again to place the $b$ billboards in proper 
	locations such that the \emph{maximum} distance of every house from some billboard is \emph{minimized}. 
	
	Design an algorithm for this problem that given the locations of the houses in the array $A[1:n]$ and the budget $b$, outputs an assignment of these billboards to the locations so that the maximum distance, among all houses, from their closest billboard is 
	minimized.  
	
\emph{Example:} If the houses are placed in $[1,5,2,7,11]$ and $b=2$, we can place one billboard at location $3$ and another at location $9$ and the answer would be $2$ -- every house is within distance $2$ of some billboard. 
For the same houses, if instead we have $b=1$, we can place a billboard in location $6$ and have the distance of at most $5$ for every house.  
\end{challenge}

\end{document}





